\documentclass[margin, 10pt]{res} % Use the res.cls style, the font size can be changed to 11pt or 12pt here

\usepackage{helvet} % Default font is the helvetica postscript font
%\usepackage{newcent} % To change the default font to the new century schoolbook postscript font uncomment this line and comment the one above

\setlength{\textwidth}{5.1in} % Text width of the document

\makeatletter
\newlength{\bibhang}
\setlength{\bibhang}{1em} %1em}
\newlength{\bibsep}
 {\@listi \global\bibsep\itemsep \global\advance\bibsep by\parsep}
\newenvironment{bibsection}%
        {\begin{enumerate}{}{%
%        {\begin{list}{}{%
       \setlength{\leftmargin}{\bibhang}%
       \setlength{\itemindent}{-\leftmargin}%
       \setlength{\itemsep}{\bibsep}%
       \setlength{\parsep}{\z@}%
        \setlength{\partopsep}{0pt}%
        \setlength{\topsep}{0pt}}}
        {\end{enumerate}\vspace{-.6\baselineskip}}
%        {\end{list}\vspace{-.6\baselineskip}}
\makeatother
\begin{document}

%----------------------------------------------------------------------------------------
%	NAME AND ADDRESS SECTION
%----------------------------------------------------------------------------------------

\moveleft.5\hoffset\centerline{\large\bf Bhawana Chhaglani} % Your name at the top
 
\moveleft\hoffset\vbox{\hrule width\resumewidth height 1pt}\smallskip % Horizontal line after name; adjust line thickness by changing the '1pt'
 

\moveleft.5\hoffset\centerline{Phone No. : +91 9958190901}
\moveleft.5\hoffset\centerline{Email : chhaglani.bhawana@gmail.com  t-bhchha@microsoft.com}
LinkedIn : \href{www.linkedin.com/in/bhawana-chhaglani-579873145}\\
Website : \href{sites.google.com/view/bhawanachhaglani}
%----------------------------------------------------------------------------------------

\begin{resume}

%----------------------------------------------------------------------------------------
%	Interests
%----------------------------------------------------------------------------------------
 
\section{INTERESTS}  

Robotics, Mobile computing, Image Processing, Internet of things and communication.

%----------------------------------------------------------------------------------------
%	Education 
%----------------------------------------------------------------------------------------
\section{EDUCATION} 
{\sl \bf{Bharati Vidyapeeth's College of Engineering}} \hfill 2016-2020\\ Bachelor of Technology in Electronics and Communication\\
Current CGPA Score - 9.0/10.0

{\sl \bf{Tulip Public School, Etah}}
\hfill Jul.2016 \\12th C.B.S.E.  \\District Topper with 95\%

{\sl \bf{Tulip Public School, Etah}}
\hfill Jul. 2014 \\10th C.B.S.E. 10 CGPA\ 

%----------------------------------------------------------------------------------------
%	Training and Internships 
%----------------------------------------------------------------------------------------

\section{TRAINING \& \\ INTERNSHIPS}

{\sl \bf{Research Intern at Microsoft Research Lab India}}\hfill Jun. 2019-Present\\
Led by : Srinivasan Iyengar(Microsoft Research) and T V Prabhakar(IISc Banagalore)


{\sl \bf{Full Stack Developer at Aerogram}}\hfill Dec. 2018-Present\\
Handling the server side of the IoT network. \\
Led by : Sarita Ahlawat

{\sl \bf{Research Intern at IIT Delhi}}\hfill Jun. 2018-Aug. 2018\\
Implementing MAC Layer for Visible Light Communication \\
Led by : Dr. Brejesh Lall(IIT Delhi), Shivam Shekhar(Velmenni) and Monica Bhutani(Bharati Vidyapeeth's College of Engineering)

{\sl \bf{Udemy Online Course}}\hfill Dec. 2017-Jan. 2018\\
Machine Learning and Deep Learning using Python


{\sl \bf{Trainee at Cyborg Labs, Delhi}}\hfill Jun. 2017-Jul. 2017\\
Embedded Systems and its Applications
%----------------------------------------------------------------------------------------
%	Technical Skills 
%----------------------------------------------------------------------------------------

\section{TECHNICAL \\ SKILLS} 

{\sl Programming Languages:} Python, C/C++, Embedded C, MATLAB, LaTex, HTML/CSS, Javascript, PHP, SQL.  \\
{\sl Hardware Descriptive Language:} VHDL. \\
{\sl Embedded Platforms:} Raspberry Pi, Arduino, NodeMCU. \\
{\sl Softwares and Libraries:} OpenCV, TensorFlow, Keras, Firebase, ATMEL AVR, Proteus, OrCAD Capture/PSpice, Firebase. 


%----------------------------------------------------------------------------------------
%	Awards 
%---------------------------------------------------------------------------------------

\section{HONORS \& AWARDS}


{\sl \bf{First Position in eYantra 2018} } \hfill Apr. 2019\\It is a national-level Robotics Competition organised by \bf{MHRD} and \bf{IIT Bombay}.

{\sl \bf{Winner of Smart India Hackathon 2019} } \hfill {\normalfont Mar. 2019\\It was a national-level competition organized by \bf{All India Council for Technical Education (AICTE), Inter Institutional Inclusive Innovation Center (i4c),} and \bf{Persistent Systems}}.

{\sl \bf{Winner of Hack-A-BIT} } \hfill Oct. 2018\\ {\normalfont First position in East-India's Largest Hackathon.}

{\sl \bf{Hackathon 5.0 Digifest Bikaner} } \hfill Jun. 2018 \\ {\normalfont Selected in top 5 teams out of 800 teams from over 20 cities.}

{\sl 2nd Runner-up in \bf{Smart India Hackathon 2018} } \hfill{\normalfont  Dec. 2017-Jun. 2018 \\ It was a national-level competition organized by \bf{All India Council for Technical Education (AICTE), Inter Institutional Inclusive Innovation Center (i4c),} and \bf{Persistent Systems}.}

{\sl First Prize at  \bf{EV Hackathon}}\hfill {\normalfont Apr. 2018\\ It was } \bf{India-Australia Joint Initiative in collaboration with Charge Point, Mahindra and BSES.}

{\sl Participation in {\bf eYantra 2017}}\hfill {\normalfont Jan. 2018\\ eYantra is a national-level Robotics Competition. The team came in top 10 out of 150 teams in theme planter bot} 


{\sl {\bf IEEE Xtreme 11.0}\hfill{\normalfont Oct. 2017\\ Xtreme is an International level Coding Competition. The team had an {\bf AIR 19} }}

{\sl {\bf Third Rank in RoboCup, DTU}
\hfill {\normalfont Feb. 2017}}

{\sl {\bf First Rank in Robosoccer, BVEST at BVCOE}
\hfill{\normalfont Oct. 2016 }}

{\sl {\bf Student of the Year Award}\hfill{\normalfont Jul. 2016\\ For standing first at district-level in CBSE Board Examinations }}
%----------------------------------------------------------------------------------------
%	Publications and Projects 
%----------------------------------------------------------------------------------------




\section{PROJECTS}

- {Pollinator Bee}
{\sl \hfill {\normalfont Sep. 2018-Apr. 2019}\\ This project involves a PlutoX drone(bee) that navigates through the desired way-points in an arena and pollinates the flower at every way-point. It involves ROS, image processing and PID.}

- {DroneVLC}
{\sl \hfill {\normalfont Nov. 2018-Present}\\ Drone to Drone Communication using visible light communication.}\\Supervised by: Ashok Ashwin(Georgia State University)

- {THE HAND}
{\sl \hfill {\normalfont Oct. 2018}\\ An Assistant for Deaf and Mute.
It consists of potentiometers placed at every joint of hand that converts sign language into speech and provides alerts in case of emergency.}

- {WieAssist}
{\sl \hfill {\normalfont Oct. 2018}\\ An application that incorporates safety in daily life of women. 
It rates places on the basis of safety using crowd-sourcing and web crawling, provides timely alerts and shows safe routes. It was built using Python, Firebase and Java.}



- {\bf Touristant} 
{\sl \hfill {\normalfont Jun. 2018-Jul. 2018}\\ A positioning-system that guides and assists tourists at tourist destinations using Received Signal Strength Indicator triangulation mechanism. }

- {\bf {MAC Layer for VLC(IIT Delhi) }  }
 { \hfill \normalfont Jun. 2018-Aug. 2018}\\{\sl MATLAB Simultion of MAC layer for Visible Light Communication based on standard IEEE802.15.7}

- {\bf Baggage Tracking at Airports}
{\sl  \hfill {\normalfont Dec. 2017-Jun. 2018 }\\A system that tracks the baggages at airport using RFID and IoT technology. It aims at strengthening the existing barcode dependent system. }

- {\bf Planter Bot(eYRC 2017)} 
 {\sl \hfill {\normalfont Nov. 2017-Feb. 2018}\\A bot that traverses an area using raspberry pi and picam and responses to various colors and shapes in that area. It follows the line using Image Processing and PID.}

- {\bf Smart Farming Assistant}
{\sl \hfill{\normalfont Aug. 2017-Mar. 2018}\\ A sytem that makes farming crop-oriented. It sows seeds, analyses the field sectionally and accordingly provides water and fertilizers.}


- {\bf Visible Light Communication}
{\sl \hfill{\normalfont Aug. 2017-Sep. 2017}\\Transfer text, music and images from one device to another using LED and Solar Panel. }

- {\bf Racing drone}
{\sl  \hfill{\normalfont Dec. 2016}\\A drone made using arduino ATMEGA 2560.}

- {\bf Advanced Line Follower Bot} 
{\sl \hfill{\normalfont Oct. 2016}\\ A line follower and color-responsive robot made using LDR and color sensor.}


%----------------------------------------------------------------------------------------
%	Leadership Experiences 
%----------------------------------------------------------------------------------------

\section{LEADERSHIP EXPERIENCE}
{\sl {\bf Head Publications and Documentations}\hfill {\normalfont Aug. 2017-Aug. 2018 \\
Computer Science Society, BVP-IEEE Student Branch, Delhi.}

{\sl {\bf Web Administrative Executive}\hfill {\normalfont Aug. 2017-Aug. 2018 \\
Robotics and Automation Society, BVP-IEEE Student Branch, Delhi.}

{\sl {\bf Event Manager at Fervour}\hfill{\normalfont Mar. 2018 \\
Organised by BVP-IEEE Student Branch, Delhi.}

{\sl {\bf Student Volunteer}} \hfill{\normalfont Aug. 2016-Aug. 2017\\ BVP-IEEE Student Branch, Delhi.}

{\sl {\bf Head Girl}}\hfill{\normalfont Jul. 2015-Jul. 2016\\
Tulip Public School, Etah}



%----------------------------------------------------------------------------------------
%	Hobbies
%----------------------------------------------------------------------------------------
\section{PUBLICATIONS AND PATENTS}
\begin{bibsection}
\item \textbf{B. Chhaglani} "Baggage Handling and Tracking at Airports" 2018. Patented at \emph{Intellectual Property India}.
\end{bibsection}

\begin{bibsection}
\item \textbf{B. Chhaglani} "An Innovative Passive Localization and Navigtion Approach"  \emph{IndiaCom 2018}.
\end{bibsection}
\section{MENTORS}


{\sl\\ {\bf Ashwin Ashok} \hfill Assistant prof. Georgia State University\\email - aashok@gsu.edu
}
{\sl\\ {\bf Srinivasan Iyengar} \hfill Microsoft Research India\\email - t-sriyen@microsoft.com
}\\
{\sl {\bf Abhishek Gagneja } \hfill Assistant Professor, BVCOE
\\ Contact No.- +91 9971122557}
{\sl\\ {\bf Monica Bhutani} \hfill Assistant Professor, BVCOE
\\ Contact No.- +91 9868344002}
{\sl\\ {\bf Sarita Ahlawat} \hfill BIRAC BIG Innovator, IIT-Delhi
\\ Contact No.- +91 9711491975}
}
\end{resume}
\end{document}
